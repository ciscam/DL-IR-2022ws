% ---------------------------------------------------------------------------- %
% -------------------------- nevermind the preamble -------------------------- %
% ---------------------------------------------------------------------------- %
\documentclass[10pt,a4paper]{article}
\usepackage[a4paper, total={160mm, 240mm}]{geometry} % smaller margin %
\usepackage[utf8]{inputenc} % umlauts %
\usepackage[german]{babel} % validation %
\usepackage[autostyle]{csquotes}
\MakeOuterQuote{"}


\usepackage{hyperref}

\usepackage{enumitem}
\setlist[enumerate, 1]{label=(\alph*)}
\setlist[enumerate, 2]{label=\arabic*}

\usepackage{titlesec}
\titleformat{\section}[hang]{\small}{\textbf{Task \thesection:}}{.5em}{}

\usepackage{amsmath}
\usepackage{amssymb}

\usepackage{graphicx}
\usepackage[export]{adjustbox}

\usepackage{ulem}

% ---------------------------------------------------------------------------- %
% -------------------------- nevermind the preamble -------------------------- %
% ---------------------------------------------------------------------------- %

\title{ \vspace{-3em}
        Assignment 05\\
		\small{\bf Digital Libraries and Foundations of Information Retrieval}\\
		\small{Winter semester 2022}}
\author{\small{1542011 Franka Brunen}, \small{1365848 Andreas Schneider}}
\date{}
\begin{document}
\setlength{\parskip}{6pt} % Remove paragraph first line h-indent and instead add v-margin %
\setlength{\parindent}{0pt}

\leftskip=1cm\rightskip=0.5cm % Indent paragraphs for readability %
\setlist[1]{leftmargin=2cm,rightmargin=0.5cm} % Indent first level of lists accordingly %

\maketitle


\section{\hfill Thesauri\hfill 2+5+2+3+3 Points}
\begin{enumerate}
    \item \begin{enumerate}
            \item S: (n) queen (the only fertile female in a colony of social insects such as bees and ants and termites; its function is to lay eggs)
            \item S: (n) queen, queen regnant, female monarch (a female sovereign ruler)
            \item S: (n) queen (the wife or widow of a king)
            \item S: (n) queen (something personified as a woman who is considered the best or most important of her kind)         \item "Paris is the queen of cities"; "the queen of ocean liners"
            \item S: (n) king, queen, world-beater (a competitor who holds a preeminent position)
            \item S: (n) fagot, faggot, fag, fairy, nance, pansy, queen, queer, poof, poove, pouf (offensive term for a homosexual man)
            \item S: (n) queen (one of four face cards in a deck bearing a picture of a queen)
            \item S: (n) queen ((chess) the most powerful piece)
            \item S: (n) queen, queen mole rat (an especially large mole rat and the only member of a colony of naked mole rats to bear offspring which are sired by only a few males)
            \item S: (n) tabby, queen (female cat) 
        \end{enumerate}
    \item \begin{enumerate}[label={(\arabic*)}]
            \item Meaning 8
            \item Meaning 2
            \item Meaning 4, 5, 6
            \item Meaning 1
            \item Meaning 9
            \item None
        \end{enumerate}
        Sometimes a thesaurus has more than one distinct meaning to a word, like in sentence (3). Also entities, like "The queen" (of England) in sentence (2) and "Queen" (band) in sentence (6) are not covered in thesauri.
    \item Meanings 2 and 3 are handled in \textit{Monarchy}, 8 and 9 in \textit{Gaming}, 1 and 11 in \textit{Science}, and 7 in \textit{Other uses}.
    \item That approach would always assign the same meaning to all occurrences of a word. Many words have very different meanings depending on the context. The most frequent meaning also has a high probability of being one of the more generic ones. Even if "run" was identified as a verb, the approach would mismatch either "The company is running fine", or "The company has not been run down".
    \item The assignment could be improved by involving context. Just like a "John Doe" is only recognizable as a distinct author when including co-authors, locations, publishers, time frames etc. in a disambiguation, the assignment of term occurrences to thesaurus entries will become less ambiguous when including semantic context. For example if "Queen" is referred to as "she", the music band is already ruled out.
\end{enumerate}


\newpage
\section{\hfill String Distances\hfill 2+4+9 Points}
\begin{enumerate}
    \item The Hamming Distance is the amount of characters which would need to be replaced to transform one string into the other. Both strings need to have the same length.\footnote{\url{https://cgl.ethz.ch/teaching/former/infotheory0607/Downloads/hamming1_1.pdf}}
    \item The Levenshtein Distance is a measure of how many single-character edit operations are required to transform one string into the other. Operations are \textit{insert}, \textit{delete} and \textit{substitute}. The variant by Damerau adds the operation of \textit{transposition}, the exchange of two neighboring characters, at the same cost as the other operations in its regular form.\footnote{\url{https://suw.biblos.pk.edu.pl/resources/i5/i0/i8/i0/i3/r50803/NiewiarowskiA_ParallelizationLevenshtein.pdf}}
    \item The Levenshtein Distances in this exercise are unweighted.
    
        \textbf{rhein} and \textbf{rhine}\\
        \begin{tabular}{llll}
             Hamming Distance               &rhein&&0\\
                                            &rh\textbf{i}in &replace \textbf{e} with \textbf{i} & 1\\
                                            &rhe\textbf{n}n &replace \textbf{i} with \textbf{n} & 2\\
                                            &rhin\textbf{e} &replace \textbf{n} with \textbf{e} & \textbf{3}\\
             Levenshtein Distance           &rhein&&0\\
                                            &rh\sout{e}in &delete \textbf{e} & 1\\
                                            &rh\ \ in\textbf{e} &insert \textbf{e} & \textbf{2}\\
             Damerau-Levenshtein Distance   &no difference\\
        \end{tabular}
    
        \textbf{mainz} and \textbf{minze}\\
        \begin{tabular}{llll}
             Hamming Distance               &mainz&&0\\
                                            &m\textbf{i}inz &replace \textbf{a} with \textbf{i} & 1\\
                                            &mi\textbf{n}nz &replace \textbf{i} with \textbf{n} & 2\\
                                            &min\textbf{z}z &replace \textbf{n} with \textbf{z} & 3\\
                                            &minz\textbf{e} &replace \textbf{z} with \textbf{e} & \textbf{4}\\
             Levenshtein Distance           &mainz&&0\\
                                            &m\sout{a}inz &delete \textbf{a}& 1\\
                                            &m\ \ inz\textbf{e} &insert \textbf{e}& \textbf{2}\\
             Damerau-Levenshtein Distance   &no difference\\
        \end{tabular}
    
        \textbf{abababab} and \textbf{babababa}\\
        \begin{tabular}{llll}
             Hamming Distance               &abababab&&0\\
                                            &\textbf{b}bababab &replace \textbf{a} with \textbf{b} & 1\\
                                            &b\textbf{a}ababab &replace \textbf{b} with \textbf{a} & 2\\
                                            &...\\
                                            &bababab\textbf{a} &replace \textbf{b} with \textbf{a} & \textbf{8}\\
             Levenshtein Distance           &abababab&&0\\
                                            &\sout{a}bababab &delete \textbf{a}& 1\\
                                            &\ \ bababab\textbf{a} &insert \textbf{a}& 1\\
             Damerau-Levenshtein Distance   &no difference\\
        \end{tabular}
\end{enumerate}


\newpage
\section{\hfill Context-sensitive Spelling Correction\hfill 2+8+5 Points}
\begin{enumerate}
    \item $ P(miles|thousand), P(males|thousand), P(mules|thousand) $
    \item Using Firefox, private window, google.com. The variable $C$ (C is the number of distinct bigrams in the corpus with $w_i$ at the first position) is left out for the purpose of simplicity.
    
        About 2.610.000.000 results for \texttt{thousand}.
        
        About 13.300.000 results for "\texttt{thousand miles}". $P(miles|thousand)=\frac{13.3\textsc{e}6}{26.1\textsc{e}8}=0.51\%$
        
        About 61.800 results for "\texttt{thousand males}". $P(males|thousand)=\frac{61800}{26.1\textsc{e}8}=0.0024\%$
        
        About 9.200 results for "\texttt{thousand mules}". $P(mules|thousand)=\frac{9200}{26.1\textsc{e}8}=0.00035\%$
        
        \texttt{miles} is the best correction term out of the \textit{three} candidates.
    \item Possible information need: "Does the ZIMK host an event for christmas?"
    
        \texttt{zimk}, as an abbreviation, could be misinterpreted as a human spelling error.
        
        Both duckduckgo.com and google.com included results for \texttt{zink} and \texttt{zinc}. Google announced it ("Did you mean \textit{zink} christmas?"), Duckduckgo did not.
        
        I am pretty sure that more people will have mistyped \texttt{zink/zinc} in this query "\texttt{zimk christmas}", looking for decoration, cups etc., than people who actually want to know about the christmas plans of whatever \texttt{zimk} means to them. That means that more than half of the time, people querying for "\texttt{zimk christmas}" are helped by including search results with the most probable correction terms.
\end{enumerate}


\end{document}