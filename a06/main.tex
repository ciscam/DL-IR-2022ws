% ---------------------------------------------------------------------------- %
% -------------------------- nevermind the preamble -------------------------- %
% ---------------------------------------------------------------------------- %
\documentclass[10pt,a4paper]{article}
\usepackage[a4paper, total={160mm, 240mm}]{geometry} % smaller margin %
\usepackage[utf8]{inputenc} % umlauts %
\usepackage[german]{babel} % validation %
\usepackage[autostyle]{csquotes}
\MakeOuterQuote{"}


\usepackage{hyperref}

\usepackage{enumitem}
\setlist[enumerate, 1]{label=(\alph*)}
\setlist[enumerate, 2]{label=\arabic*}

\usepackage{titlesec}
\titleformat{\section}[hang]{\small}{\textbf{Task \thesection:}}{.5em}{}

\usepackage{amsmath}
\usepackage{amssymb}

\usepackage{graphicx}
\usepackage[export]{adjustbox}

% ---------------------------------------------------------------------------- %
% -------------------------- nevermind the preamble -------------------------- %
% ---------------------------------------------------------------------------- %

\title{ \vspace{-3em}
        Assignment 06\\
		\small{\bf Digital Libraries and Foundations of Information Retrieval}\\
		\small{Winter semester 2022}}
\author{\small{1542011 Franka Brunen}, \small{1365848 Andreas Schneider}}
\date{}
\begin{document}
\setlength{\parskip}{6pt} % Remove paragraph first line h-indent and instead add v-margin %
\setlength{\parindent}{0pt}

\leftskip=1cm\rightskip=0.5cm % Indent paragraphs for readability %
\setlist[1]{leftmargin=2cm} % Indent first level of lists accordingly %

\maketitle

\section{\hfill Vector Space Model\hfill 2+3+3+2+1+4 Points}
\begin{enumerate}
    \item \begin{tabular}[t]{ccc}
             $idf_{t_1} = $&$\log\frac{4}{2}$ & $=1$  \\
             $idf_{t_2} = $&$\log\frac{4}{4}$ & $=0$  \\
             $idf_{t_3} = $&$\log\frac{4}{2}$ & $=1$  \\
             $idf_{t_4} = $&$\log\frac{4}{2}$ & $=1$  \\
             $idf_{t_5} = $&$\log\frac{4}{1}$ & $=2$  \\
             $idf_{t_6} = $&$\log\frac{4}{2}$ & $=1$  \\
             $idf_{t_7} = $&$\log\frac{4}{1}$ & $=2$
        \end{tabular}
    \item \begin{tabular}[t]{lcccc}
             & $d_1$&$d_2$&$d_3$&$d_4$ \\
            biscuits $(t_1)$    & $0$&$3$&$0$&$4$ \\
            stollen $(t_2)$    & $0$&$0$&$0$&$0$ \\
            lebkuchen $(t_3)$    & $1$&$7$&$0$&$0$ \\
            macaroons $(t_4)$    & $0$&$1$&$0$&$2$ \\
            brownies $(t_5)$    & $0$&$2$&$0$&$0$ \\
            cookies $(t_6)$    & $0$&$1$&$0$&$4$ \\
            pastries $(t_7)$    & $0$&$0$&$8$&$0$ \\
        \end{tabular}
    \item \begin{tabular}[t]{lcccc}
             & $d_1$&$d_2$&$d_3$&$d_4$ \\
            biscuits $(t_1)$    & $0$&$\frac{3}{8}$&$0$&$\frac{4}{6}$ \\
            stollen $(t_2)$    & $0$&$0$&$0$&$0$ \\
            lebkuchen $(t_3)$    & $1$&$\frac{7}{8}$&$0$&$0$ \\
            macaroons $(t_4)$    & $0$&$\frac{1}{8}$&$0$&$\frac{2}{6}$ \\
            brownies $(t_5)$    & $0$&$\frac{2}{8}$&$0$&$0$ \\
            cookies $(t_6)$    & $0$&$\frac{1}{8}$&$0$&$\frac{4}{6}$ \\
            pastries $(t_7)$    & $0$&$0$&$1$&$0$ \\
        \end{tabular}
    \item $q=(0,0,\frac{1}{\sqrt{3}},0,0,\frac{1}{\sqrt{3}},\frac{1}{\sqrt{3}})$
    \item Summing up the normalized tf-idf scores on query terms, documents 1 through 3 will score either identical on a highest sum of 1, or document 2 will score better/worse, since its sum is distributed.
    \item \begin{tabular}[t]{ccccccccc}
            $sim(d_1,q)=$&0&+ 0&+ $1*\frac{1}{\sqrt{3}}$&+ 0&+ 0&+ 0&+ 0&= $1*\frac{1}{\sqrt{3}}$\\
            $sim(d_2,q)=$&0&+ 0&+ $\frac{7}{8}*\frac{1}{\sqrt{3}}$&+ 0&+ 0&+ $\frac{1}{8}*\frac{1}{\sqrt{3}}$&+ 0&= $1*\frac{1}{\sqrt{3}}$\\
            $sim(d_3,q)=$&0&+ 0&+ 0&+ 0&+ 0&+ 0&+ $1*\frac{1}{\sqrt{3}}$&= $1*\frac{1}{\sqrt{3}}$\\
            $sim(d_4,q)=$&0&+ 0&+ 0&+ 0&+ 0&+ $\frac{4}{6}*\frac{1}{\sqrt{3}}$&+ 0&= $\frac{4}{6}*\frac{1}{\sqrt{3}}$
        \end{tabular}
        
        Having equal weights, it did not make a difference that the sum is distributed.
\end{enumerate}


\section{\hfill Vector Space Model\hfill 1+2+3+4+5 Points}
\begin{enumerate}
    \item It is irrelevant except if the query consists of a term with $idf=0$. Then the query will return all documents equally, instead of a ranking by term frequency.
    \item \begin{tabular}[t]{cccccccc}
             $|d_1|$&$||d_1||$&$|d_2|$&$||d_2||$&$|d_3|$&$||d_3||$&$|d_4|$&$||d_4||$\\
             =27&=$\sqrt{427}$&=27&=$\sqrt{269}$&=105&=$\sqrt{10025}$&=1015&=$\sqrt{1000125}$
        \end{tabular}
    \item $sim(d_1,q)=\frac{1}{\sqrt{427}}$\\
        $sim(d_2,q)=\frac{10}{\sqrt{269}}$\\
        $sim(d_3,q)=0$\\
        $sim(d_4,q)=0$
    \item $sim(d_1,q)=\frac{5}{\sqrt{427}}$\\
        $sim(d_2,q)=\frac{5}{\sqrt{269}}$\\
        $sim(d_3,q)=\frac{5}{\sqrt{10025}}$\\
        $sim(d_4,q)=\frac{5}{\sqrt{1000125}}$
        
        The documents do not have the same score since each document vector has a different length. It could be reasonable to believe, that a document is more relevant if the relative frequency of the query term is higher, even though the absolute frequency is lower, e.g. one paragraph about the term alone versus in a long text (including other topics).
    \item $sim(d_1,q)=\frac{1}{\sqrt{427}}$\\
        $sim(d_2,q)=0$\\
        $sim(d_3,q)=0$\\
        $sim(d_4,q)=\frac{10}{\sqrt{1000125}}$
        
        In task (c), document 1 was less relevant than the other document, document 2. In this task, document 1 is more relevant than the other document, document 4.
        
        This is because document 4 is shorter than document 2, making the identical absolute number of occurrences less impactful.
\end{enumerate}


\end{document}