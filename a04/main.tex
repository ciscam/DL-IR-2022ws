% ---------------------------------------------------------------------------- %
% -------------------------- nevermind the preamble -------------------------- %
% ---------------------------------------------------------------------------- %
\documentclass[10pt,a4paper]{article}
\usepackage[a4paper, total={160mm, 240mm}]{geometry} % smaller margin %
\usepackage[utf8]{inputenc} % umlauts %
\usepackage[german]{babel} % validation %
\usepackage[autostyle]{csquotes}
\MakeOuterQuote{"}


\usepackage{hyperref}

\usepackage{enumitem}
\setlist[enumerate, 1]{label=(\alph*)}
\setlist[enumerate, 2]{label=\arabic*}

\usepackage{titlesec}
\titleformat{\section}[hang]{\small}{\textbf{Task \thesection:}}{.5em}{}

\usepackage{amsmath}
\usepackage{amssymb}

\usepackage{graphicx}
\usepackage[export]{adjustbox}

\usepackage[table,xcdraw]{xcolor}

% ---------------------------------------------------------------------------- %
% -------------------------- nevermind the preamble -------------------------- %
% ---------------------------------------------------------------------------- %

\title{ \vspace{-3em}
        Assignment 04\\
		\small{\bf Digital Libraries and Foundations of Information Retrieval}\\
		\small{Winter semester 2022}}
\author{\small{1542011 Franka Brunen}, \small{1365848 Andreas Schneider}}
\date{}
\begin{document}
\setlength{\parskip}{6pt} % Remove paragraph first line h-indent and instead add v-margin %
\setlength{\parindent}{0pt}

\leftskip=1cm\rightskip=0.5cm % Indent paragraphs for readability %
\setlist[1]{leftmargin=2cm} % Indent first level of lists accordingly %

\maketitle

\section{\hfill Boolean Retrieval\hfill 3+3+4+3+2 Points}
\begin{enumerate}
    \item \begin{tabular}[t]{lcccc}
         & $D_1$& $D_2$& $D_3$& $D_4$ \\
         christmas&1&1&1&1\\
         snow&1&0&1&0\\
         skating&1&0&0&0\\
         wine&1&1&0&0\\
         skiing&0&1&0&0\\
         ropeway&0&1&0&0\\
         punch&0&1&0&0\\
         cross-country&0&0&1&0\\
         ice&0&0&1&1\\
         strudel&0&0&0&1\\
         cinnamon&0&0&0&1
    \end{tabular}
    \item \begin{tabular}[t]{
        >{\columncolor[HTML]{5B9BD5}}l 
        >{\columncolor[HTML]{5B9BD5}}l l
        >{\columncolor[HTML]{5B9BD5}}l lll}
            {\color[HTML]{FFFFFF} \textbf{christmas}}     & {\color[HTML]{FFFFFF} \textbf{4}} & \textbf{$\rightarrow$} & {\color[HTML]{FFFFFF} \textbf{1}} & \cellcolor[HTML]{5B9BD5}{\color[HTML]{FFFFFF} \textbf{2}} & \cellcolor[HTML]{5B9BD5}{\color[HTML]{FFFFFF} 3} & \cellcolor[HTML]{5B9BD5}{\color[HTML]{FFFFFF} 4} \\
            {\color[HTML]{FFFFFF} \textbf{snow}}          & {\color[HTML]{FFFFFF} \textbf{2}} & \textbf{$\rightarrow$} & {\color[HTML]{FFFFFF} \textbf{1}} & \cellcolor[HTML]{5B9BD5}{\color[HTML]{FFFFFF} \textbf{3}} &                                                  &                                                  \\
            {\color[HTML]{FFFFFF} \textbf{skating}}       & {\color[HTML]{FFFFFF} \textbf{1}} & \textbf{$\rightarrow$} & {\color[HTML]{FFFFFF} \textbf{1}} & \textbf{}                                                 &                                                  &                                                  \\
            {\color[HTML]{FFFFFF} \textbf{wine}}          & {\color[HTML]{FFFFFF} \textbf{2}} & \textbf{$\rightarrow$} & {\color[HTML]{FFFFFF} \textbf{1}} & \cellcolor[HTML]{5B9BD5}{\color[HTML]{FFFFFF} \textbf{2}} &                                                  &                                                  \\
            {\color[HTML]{FFFFFF} \textbf{skiing}}        & {\color[HTML]{FFFFFF} \textbf{1}} & \textbf{$\rightarrow$} & {\color[HTML]{FFFFFF} \textbf{2}} & \textbf{}                                                 &                                                  &                                                  \\
            {\color[HTML]{FFFFFF} \textbf{ropeway}}       & {\color[HTML]{FFFFFF} \textbf{1}} & \textbf{$\rightarrow$} & {\color[HTML]{FFFFFF} \textbf{2}} & \textbf{}                                                 &                                                  &                                                  \\
            {\color[HTML]{FFFFFF} \textbf{punch}}         & {\color[HTML]{FFFFFF} \textbf{1}} & \textbf{$\rightarrow$} & {\color[HTML]{FFFFFF} \textbf{2}} & \textbf{}                                                 &                                                  &                                                  \\
            {\color[HTML]{FFFFFF} \textbf{cross-country}} & {\color[HTML]{FFFFFF} \textbf{1}} & \textbf{$\rightarrow$} & {\color[HTML]{FFFFFF} \textbf{3}} & \textbf{}                                                 &                                                  &                                                  \\
            {\color[HTML]{FFFFFF} \textbf{ice}}           & {\color[HTML]{FFFFFF} \textbf{2}} & \textbf{$\rightarrow$} & {\color[HTML]{FFFFFF} \textbf{3}} & \cellcolor[HTML]{5B9BD5}{\color[HTML]{FFFFFF} \textbf{4}} &                                                  &                                                  \\
            {\color[HTML]{FFFFFF} \textbf{strudel}}       & {\color[HTML]{FFFFFF} \textbf{1}} & \textbf{$\rightarrow$} & {\color[HTML]{FFFFFF} \textbf{4}} & \textbf{}                                                 &                                                  &                                                  \\
            {\color[HTML]{FFFFFF} \textbf{cinnamon}}      & {\color[HTML]{FFFFFF} \textbf{1}} & \textbf{$\rightarrow$} & {\color[HTML]{FFFFFF} \textbf{4}} & \textbf{}                                                 &                                                  &          
        \end{tabular}
    \item \begin{enumerate}
            \item evaluate \textit{(ice \textbf{OR} punch)}
                \begin{enumerate}
                    \item t1 = "ice"
                    \item p1 = \{3, 4\}
                    \item t2 = "punch"
                    \item p2 = \{2\}
                    \item p1 $\cup$ p2 = \{2, 3, 4\}
                \end{enumerate}
            \item evaluate \textit{christmas \textbf{AND} (ice \textbf{OR} punch)}
                \begin{enumerate}
                    \item t1 = "christmas"
                    \item p1 = \{1, 2, 3, 4\}
                    \item t2 = \textit{(ice \textbf{OR} punch)}
                    \item p2 = \{2, 3, 4\}
                    \item p1 $\cap$ p2 = \{2,3,4\}
                \end{enumerate}
        \end{enumerate}
    \item Information need: \textit{Should I do sport on the ice or drink punch to warm up on christmas?}\\
        Relevant documents: $D_1$ (skating on ice), $D_2$, $D_3$, $D_4$\\
        Precision: 3/3\\
        Recall: 3/4\\
        Recall was not perfect, as I added $D_1$ by my understanding of the term \textit{ice} as in \textit{skating on ice}. The Boolean Query Algorithm does not account for such far fetched relations and thus can not find this explicit-positive document.
    \item The term christmas appears in all documents. That means it either does not have any effect on a query, or, when disjunctive, always returns all documents, or, when negated and conjunctive, the result set is always empty.
\end{enumerate}


\section{\hfill Tokenisation\hfill 3+12 Points}
\begin{enumerate}
    \item  A token is an instance of a limited character string that occurs in a given document and is grouped into a semantically meaningful unit for further processing.
    A token can occur more than once in a document. \\
    
    A Term is a (possibly „normalized“) type that is added to the vocabulary. Normalization can be done for example with respect to upper and lower case, morphology (part of speech, flection, etc.), spelling. \\
    
    \item  \textit{Punctuation marks} \\
                 During tokenization the following punctuaction marks are normally ignored: \\
                . , ; : ? ! ' ": \\
                O'Connor as \enquote{O} and \enquote{Connor} \\
                \\
           \textit{Hyphens} \\
                In the example sentence, it is not clear how \enquote{Peter-Paul-and-Mary} is tokenized. \\
                Peter-Paul-and-Mary as \enquote{Peter}, \enquote{Paul},\enquote{and} and \enquote{Mary} or \enquote{PeterPaulandMary} or \enquote{Peter-Paul-and-Mary} \\
                \\
           \textit{Umlauts}\\
                In this example sentence, the word \enquote{Kürenz} is also unclear when tokenized. \\
                Kürenz as \enquote{Kuerenz} or \enquote{Kurenz}\\
                \\
          \textit{Hyphenation at line end} \\
                In the example sentence, the word \enquote{Ceremony} is separated by a hyphen at the end of the line, making tokenization more difficult. \\
                Ceremony as \enquote{cere} and \enquote{mony} \\
\end{enumerate}  


\section{\hfill Document preprocessing\hfill 15 Points}
\begin{enumerate}
    \item \textbf{Tokens:} analyzing, online, schema, extraction, approaches, for, linked, data, knowledge, bases, elements, of, computer, science, artificial, intelligence
    \item \textbf{Stop words (to be removed):} for, of
    \item \textbf{Porter Algorithm}, where (1b-2) means \textit{Step 1b rule 2} as per the original definition\footnote{\url{https://tartarus.org/martin/PorterStemmer/def.txt}}:
    
        \begin{tabular}{r|l}
            analyzing&(1b-3) analyz\\
            online&(5a-1) onlin\\
            schema&\\
            extraction&(4-12) extract\\
            approaches&(1a-4) approache (5a-1) approach\\
            linked&(1b-2) link\\
            data&\\
            knowledge&(5a-1) knowledg\\
            bases&(1a-4) base (5a-2) bas\\
                &(\textit{nltk.stem.PorterStemmer returned "base" but b-\textbf{as} is m=1},\\
                &\textit{bas is/ends CVC and the second c is not W, X, or Y...})\\
            elements&(1a-4) element \\
            computer&(4-4) comput\\
            science&(5a-2) scienc\\
            artificial&(4-1) artifici\\
            intelligence&(4-3) intellig
        \end{tabular}
\end{enumerate}


\end{document}